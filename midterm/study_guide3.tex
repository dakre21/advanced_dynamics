\documentclass[conference]{IEEEtran}
\IEEEoverridecommandlockouts
% The preceding line is only needed to identify funding in the first footnote. If that is unneeded, please comment it out.
%Template version as of 6/27/2024

\usepackage{cite}
\usepackage{amsmath,amssymb,amsfonts}
\usepackage{algorithmic}
\usepackage{graphicx}
\usepackage{textcomp}
\usepackage{xcolor}
\def\BibTeX{{\rm B\kern-.05em{\sc i\kern-.025em b}\kern-.08em
    T\kern-.1667em\lower.7ex\hbox{E}\kern-.125emX}}
\begin{document}

\title{MCEN 5228: Advanced Dynamics Notes}

\author{David Akre}

\maketitle

\begin{abstract}
These notes will focus on Energy, Work, and Lagrangian Mechanics but include primer material on 6DoF kinematics and dynamics (i.e., Rigid Bodies).
\end{abstract}

\begin{IEEEkeywords}
    Energy, Work, Work-Energy Principle, Lagrangian Mechanics, Rigid Bodies
\end{IEEEkeywords}

\section{Rigid Bodies}
\subsection{Newtonian Mechanics}
The process to solve rigid body dynamics problems.
\begin{enumerate}
    \item Pick an origin and attach the fixed inertial coordinate frame of reference
    \item Establish moving coordinate frames to describe each body of interest w.r.t. CoM
    \item For each body of interest derive the kinematics: $\vec{r}_i$, $\dot{\vec{r}}_i$, $\ddot{\vec{r}}_i$, and angular momentum $H_i$
    \item Then draw a FBD describing the external forces $F_i$ and moments $M_i$ acting upon rigid body i
    \item Write out the balance laws and establish equations of motion (6 EQs)
\end{enumerate}

\textbf{Coordinate Frames}

\begin{itemize}
   \item \underline{Carteisan Coordinates} (i.e., fixed inertial frame $F = \{\hat{f}_x, \hat{f}_y, \hat{f}_z\}$) \begin{align*}
   & \vec{r}^P = \begin{bmatrix} x & y & z\end{bmatrix} \begin{bmatrix} \hat{f}_x & \hat{f}_y & \hat{f}_z \end{bmatrix}^T
   \end{align*}
   Note, $\frac{d}{dt}\vec{r}^P$ and $\frac{d^2}{dt^2}\vec{r}^P$ components may be non-zero but $\frac{d}{dt}F = 0$ since it is non-moving.
   \item \underline{Cylindrical Coordinates} a moving frame relative to fixed frame $K_c = \{\hat{e}_r, \hat{e}_\phi, \hat{e}_z\}$ \begin{align*}
   & \vec{r}^P = r \hat{e}_r + z \hat{e}_z \\
   & \vec{v}^P = \dot{r}\hat{e}_r + \dot{z}\hat{e}_z + r \dot{\hat{e}}_e + z \dot{\hat{e}}_z 
   \end{align*}
Time derivative of a unit verctor is $\hat{e}_i = \vec{\omega} \times \hat{e}_i$ where $\vec{\omega}$ is the angular velocity $[p, q, r]$ of a rigid body containing $\hat{e}_i$, so $\vec{\omega} = \dot{\phi} \hat{e}_z$ (i.e., $\perp$ to rotating plane):
   \begin{align*}
    & \begin{bmatrix} \dot{\hat{e}}_r \\ \dot{\hat{e}}_\phi \\ \dot{\hat{e}}_z \end{bmatrix} = \vec{\omega} \times \begin{bmatrix} \hat{e}_r \\ \hat{e}_\phi \\ \hat{e}_z \end{bmatrix} = \begin{bmatrix} \dot{\phi} \hat{e}_\phi \\ -\dot{\phi} \hat{e}_r \\ 0 \end{bmatrix}
   \end{align*}
   So $\vec{v}^P = \dot{r} \hat{e}_r + r \dot{\phi} \hat{e}_{\phi} + \dot{z}\hat{e}_z$ and $\vec{a}^P = \dot{\vec{v}}^P$ just continue to take derivative leverage the product rule for the rotating frame. The transformation between fixed and this frame is:
   \begin{align*}
    & \begin{bmatrix} \hat{f}_x \\ \hat{f}_y \\ \hat{f}_z \end{bmatrix} = \begin{bmatrix} c\theta & -s\theta & 0 \\ s\theta & c\theta & 0 \\ 0 & 0 & 1 \end{bmatrix} \begin{bmatrix} \hat{e}_r \\ \hat{e}_\phi \\ \hat{e}_z \end{bmatrix}
   \end{align*}
   \item \underline{Spherical Coordinates} another moving frame relative to the fixed frame $K_s = \{\hat{e}_r, \hat{e}_\phi, \hat{e}_\theta\}$ where $\hat{e}_r \perp \{\hat{e}_\phi, \hat{e}_{\theta}\}$.
   \begin{align*}
    & \vec{r}^P = r \hat{e}_r \text{ and } \vec{v}^P = \dot{r} \hat{e}_r + r \dot{\hat{e}}_r
   \end{align*}
   Where $\phi$ cases rotation about $\hat{e}_z$ and $\theta$ causes rotation about $\hat{e}_\phi$ so $\vec{\omega} = \dot{\phi} \hat{e}_z + \dot{\phi} \hat{e}_\phi$ where $\hat{e}_z = c\theta \hat{e}_r - s\theta \hat{e}_\theta$ so:
   \begin{align*}
    & \vec{\omega} = \dot{\phi}\cos\theta \hat{e}_r - \dot{\phi}\sin\theta\hat{e}_\theta + \dot{\theta} \hat{e}_\phi \\
    & \dot{\hat{e}}_r = \vec{\omega} \times \hat{e}_r = \dot{\theta} \hat{e}_\theta + \dot{\phi} \sin\theta \hat{e}_\phi \\
    & \dot{\hat{e}}_\theta = - \dot{\theta} \hat{e}_r + \dot{\phi} \cos\theta \hat{e}_\phi \\
    & \dot{\hat{e}}_\phi = - \dot{\phi} \sin \theta \hat{e}_r - \dot{\phi} \cos \theta \hat{e}_\theta \\
    & \vec{v}^P = \dot{r} \hat{e}_r + r \dot{\theta} \hat{e}_\theta + r \dot{\phi} \sin \theta \hat{e}_\phi
   \end{align*}
   The transformation from spherical to cartesian is the following:
   \begin{align*}
    \begin{bmatrix} \hat{f}_x \\ \hat{f}_y \\ \hat{f}_z \end{bmatrix} = \begin{bmatrix} s\theta c\phi & c\theta c\phi & -s\phi \\ s\theta s\phi & c\theta s\phi & c\phi \\ c\theta & -s\theta & 0 \end{bmatrix}\begin{bmatrix} \hat{e}_r \\ \hat{e}_\theta \\ \hat{e}_\phi \end{bmatrix}
   \end{align*}
\end{itemize}
\textbf{Newton's EoM} for rigid bodies the forces of interaction cancel so external forces must equal the inertial.
\begin{align*}
    & m_i \ddot{r}_i = F_i \quad i \in [1, ..., n]
\end{align*}
\textbf{Conservation of Momentum} rate of change of a system is euqal to the sum of all external forces acting on points of the system

\end{document}
